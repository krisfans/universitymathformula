% !TEX program = xelatex
% !TEX spellcheck
% !TEX encoding = UTF-8
% 使用 ctexart 文类,UTF-8 编码,xelatex编译
\documentclass[a4paper,zihao=-4,UTF8]{ctexbook}

\usepackage[left=3.17cm,right=3.17cm,top=2.54cm,bottom=2.54cm]{geometry}%页面设置
\usepackage{booktabs,bigstrut,multirow}%插入三线表,表格
\usepackage{enumitem}%列表
\usepackage{amsmath,amssymb,amsfonts}%数学公式
\usepackage{unicode-math}
% \usepackage{graphicx}%插图subfigure,floatrow
\usepackage{caption,subcaption}%图表标题
\captionsetup{font=small,skip=6pt,textfont=it}%修改题注间距和字体
\renewcommand{\figurename}{{\kaishu 图}}
\renewcommand{\tablename}{{\kaishu 表}}
\usepackage[breaklinks,colorlinks]{hyperref}%%目录引用等超链接
\usepackage{zhlineskip}%行距
\usepackage{zhlipsum}%中文乱数假文
\usepackage{fontspec}

\renewcommand{\emph}{\kaishu}
%\makeatletter%计数器在新的章节自动清零
%    \@addtoreset{equation}{section}
%    \@addtoreset{figure}{section}
%    \@addtoreset{table}{section}
%\makeatother
%\renewcommand {\thetable} {\thesection{}-\arabic{table}}%使表编号与章节关联
%\renewcommand {\thefigure} {\thesection{}-\arabic{figure}}%图编号
%\renewcommand {\theequation} {\thesection{}-\arabic{equation}}%公式编号

\usepackage[super]{gbt7714}%参考文献角标数字顺序编码

% \pagestyle{plain}

\title{\heiti 全国硕士研究生\\统一入学考试数学公式大全}
\author{krifan\thanks{xdu}}
\date{\small\today}
\begin{document}
\maketitle
\pagenumbering{Roman}
\tableofcontents
\newpage
\pagenumbering{arabic}
\CJKfontspec{Source Han Serif CN Light}%思源宋体
\chapter{高等数学部分}
\section{函数、极限}
\subsection{一些初等函数}
双曲正弦\[\sinh x=\frac{e^x-e^{-x}}{2}\]

双曲余弦\[\cosh x=\frac{e^x+e^{-x}}{2}\]

双曲正切\[\tanh x=\frac{\sinh x}{\cosh x}=\frac{e^x-e^{-x}}{e^x+e^{-x}}\]

反双曲正弦\[\operatorname{arcsinh} x=\ln\left(x+\sqrt{x^2+1}\right)\]

反双曲余弦\[\operatorname{arccosh} x=\pm\ln\left(x+\sqrt{x^2-1}\right)\]

反双曲正切\[\operatorname{arctanh} x=\frac12\ln\frac{1+x}{1-x}\]
\subsection{两个重要极限}
\[\lim_{x\to 0}\frac{\sin x}{x}=1\]
\[\lim_{x\to 0}(1+\frac 1x)^x=e=2.718281828459\ldots\]
\subsection{三角函数公式}
\subsubsection{和差角公式}
\begin{gather*}
    \sin(x\pm y)=\sin x\cos y\pm\cos x\sin y\\
\cos(x\pm y)=\cos x\cos y\mp\sin x\sin y\\
\tan(x\pm y)=\frac{\tan x\pm\tan y}{1\mp\tan x\tan y}
\end{gather*}
\subsubsection{积化和差公式}
\begin{align*}
    2\cos x \cos y &=\cos \left( x -y \right) +\cos \left( x +y \right) \\
    2\sin x \sin y &=\cos \left( x +y \right) -\cos \left( x +y \right) \\
    2\sin x \cos y &=\sin \left( x -y \right) +\sin \left( x +y \right) 
\end{align*}
\subsubsection{和差化积公式}
\begin{align*}
\sin x+\sin y&=2\sin\frac{x+y}{2}\cos\frac{x-y}{2}\\
\sin x-\sin y&=2\cos\frac{x+y}{2}\sin\frac{x-y}{2}\\
\cos x+\cos y&=2\cos\frac{x+y}{2}\cos\frac{x-y}{2}\\
\cos x-\cos y&=2\sin\frac{x+y}{2}\sin\frac{x-y}{2}
\end{align*}
\subsubsection{倍角公式}
\begin{gather*}
\sin 2x=2\sin x\cos x\\
\cos 2x=2\cos^2 x-1=1-2\sin^2 x=\cos^2 x-\sin^2 x\\
\tan 2x=\frac{2\tan x}{1-\tan^2 x}\\
\sin 3x=3\sin x-4\sin^3 x\\
\cos 3x=4\cos^3 x-3\cos x\\
\tan 3x=\frac{3\tan x-\tan^3 x}{1-3\tan^3 x}
\end{gather*}
\subsubsection{半角公式}
\begin{gather*}
    \sin\frac x2=\pm\sqrt{\frac{1-\cos x}{2}}\\
\cos\frac x2=\pm\sqrt{\frac{1+\cos x}{2}}\\
\tan\frac x2=\pm\sqrt{\frac{1-\cos x}{1+\cos x}}=\frac{1-\cos x}{\sin x}=\frac{\sin x}{1+\cos x}
\end{gather*}
\subsubsection{正弦定理}
\[\frac{a}{\sin A}=\frac{b}{\sin B}=\frac{c}{\sin C}=2R\ \text{($R$是外接圆半径)}\]
\subsubsection{余弦定理}
\[c^2=a^2+b^2-2ab\cos C\]
\subsubsection{反三角函数性质}
\[\arcsin x+\arccos x=\frac \pi2 \quad \arctan x+\operatorname{arccot} x=\frac \pi2\]
\section{一元函数微分学}
\subsection{导数公式}
\begin{align*}
a^x{'}&=a^x\ln x & \log_a x'&=\frac{1}{x\ln a}\\
\sin x'&=\cos x   & \cos x'&=-\sin x \\
\tan x'&=\sec^2x  & \cot x'&=-\csc^2x\\
\sec x'&=\sec x\tan x & \csc x'&=-\csc x\cot x\\
\arcsin x' &=\frac{1}{\sqrt{1-x^2}}& \arccos x' &=-\frac{1}{\sqrt{1-x^2}}\\
\arctan x' &=\frac{1}{1+x^2}&\operatorname{arccot}x'&=\frac{1}{1+x^2}
\end{align*}
\subsection{莱布尼兹(Leibniz)公式}
\[(uv)^{(n)}=\sum_{k=0}^n C_n^k u^{(n-k)}v^{k}\]
\subsection{中值定理与导数应用}
\subsubsection{拉格朗日(LaGrange)中值定理}
\[f(b)-f(a)=f'(\xi)(b-a)\]
\subsubsection{柯西(Cauchy)中值定理}
\[\frac{f(b)-f(a)}{F(b)-F(a)}=\frac{f'(\xi)}{F'(\xi)}\]

当$F(x)=x$时,柯西中值定理就是拉格朗日中值定理。
\section{一元函数积分学}

\subsection{基本不定积分表}
\subsubsection{含三角函数积分}
\begin{align*}
 \int\tan xdx&=-\ln|\cos x|+C & \int\cot xdx&=\ln|\sin x|+C\\
 \int\sec xdx&=\ln|\sec x+\tan x|+C & \int\csc xdx&=\ln|\csc x-\cot x|+C\\
\int\sec^2 xdx&=\tan x+C&\int\csc^2 xdx &=-\cot x+C\\
\int\sec x\tan xdx&=\sec x+C&\int\csc x\cot xdx&=-\csc x+C
\end{align*}
\subsubsection{含分式积分}
\begin{align*}
\int \frac{1}{a^{2}+x^{2}}dx&=\frac{1}{a} \arctan \frac{x}{a}+C &
\int \frac{1}{x^{2}-a^{2}}dx&=\frac{1}{2 a} \ln \left|\frac{x-a}{x+a}\right|+C\\
    \int \frac{1}{\sqrt{a^{2}-x^{2}}}dx&=\arcsin \frac{x}{a}+C&
    \int \frac{d x}{\sqrt{x^{2} \pm a^{2}}}&=\ln \left|x+\sqrt{x^{2} \pm a^{2}}\right|+C
\end{align*}
\subsubsection{含根式的积分}
\begin{align*}
    \int \sqrt{x^{2}+a^{2}} d x&=\frac{x}{2} \sqrt{x^{2}+a^{2}}+\frac{a^{2}}{2} \ln \left|x+\sqrt{x^{2}+a^{2}}\right|+C \\
     \int \sqrt{x^{2}-a^{2}} d x&=\frac{x}{2} \sqrt{x^{2}-a^{2}}-\frac{a^{2}}{2} \ln \left|x+\sqrt{x^{2}-a^{2}}\right|+C \\ 
     \int \sqrt{a^{2}-x^{2}} d x&=\frac{x}{2} \sqrt{a^{2}-x^{2}}+\frac{a^{2}}{2} \arcsin \frac{x}{a}+C
\end{align*}
\subsubsection{三角函数的有理式积分}
\[\sin x=\frac{2u}{1+u^2}\quad \cos x=\frac{1-u^2}{1+u^2}\quad du=\tan\frac x2\quad dx=\frac{2du}{1+u^2}\]
\subsection{定积分}
\subsubsection{三角函数的定积分}
\[
I_{n}=\int_{0}^{\frac{\pi}{2}} \sin ^{n} x d x=\int_{0}^{\frac{\pi}{2}} \cos ^{n} x d x=\frac{n-1}{n} I_{n-2}
\]
\subsubsection{换元积分}
\subsubsection{分部积分}
\subsection{曲率}
弧微分公式$ds=\sqrt{1+y'^2}dx$

平均曲率$\bar{K}=|\Delta\alpha/\Delta s|$,$\Delta\alpha $表示从$M$点到$M'$斜线斜率倾角变化量,$\Delta s$表示$MM'$弧长。

$M$点的曲率
\[K=\lim_{\Delta s\to 0}\left| \frac{\Delta \alpha}{\Delta s}\right|=\left| \frac{d \alpha}{d s}\right|=\frac{|y''|}{\sqrt{(1+y'^2)^3}}
\]
直线$K=0$,半径为$a$的圆$K= 1/a$
\subsection{定积分的近似计算}
矩形法
\[
\int_{a}^{b} f(x) \approx \frac{b-a}{n}\left(y_{0}+y_{1}+\cdots+y_{n-1}\right)
\]

梯形法
\[
\int_{a}^{b} f(x) \approx \frac{b-a}{n}\left[\frac{1}{2}\left(y_{0}+y_{n}\right)+y_{1}+\cdots+y_{n-1}\right]
\]

抛物线法
\[
\int_{a}^{b} f(x) \approx \frac{b-a}{3 n}\left[\left(y_{0}+y_{n}\right)+2\left(y_{2}+y_{4}+\cdots+y_{n-2}\right)+4\left(y_{1}+y_{3}+\cdots+y_{n-1}\right)\right]
\]
\subsection{定积分应用相关公式}
功:$W=\vec{F}\cdot\vec{s}$\quad 引力:$F=k{m_1m_2}/{r^2}$

函数的平均值:\[\bar{y}=\frac{1}{b-a}\int_a^bf(x)dx\]

均方根:\[\sqrt{\frac{1}{b-a}\int_a^b f^2(t)dt}\]
\section{空间解析几何和向量代数}
\subsection{数量积}
向量的数量积\[\vec{a}\cdot\vec{b}=|\vec{a}|\cdot|\vec{b}|\cos\theta=
a_xb_x+a_yb_y+a_zb_z\]
\subsection{向量积}
向量的向量积
\[\vec{a}\times\vec{b}=\left|\begin{array}{@{}ccc@{}}
    i&j&k\\
    a_x&a_y&a_z\\
    b_x&b_y&b_z\\
\end{array}\right|=|\vec{a}|\cdot|\vec{b}|\sin\theta\]
\subsection{混合积}
向量的混合积,$\theta$ 为锐角时,代表平行六面体的体积。
\[\left[\vec{a}\vec{b}\vec{c}\right]=(\vec{a}\times\vec{b})\cdot\vec{c}=\left|
\begin{array}{@{}lcr@{}}
    a_x&a_y&a_z\\
    b_x&b_y&b_z\\
    c_x&c_y&c_z\\
\end{array}
\right|=|\vec{a}\times\vec{b}|\cdot|\vec{c}|\sin\theta\]
\subsection{向量夹角}
设向量$\vec{a}=(a_x,a_y,a_z)$,$\vec{b}=(b_x,b_y,b_z)$
则两个向量之间的夹角:
\[\cos \theta=\frac{a_xb_x+a_yb_y+a_zb_z}{\sqrt{a_x^2+a_y^2+a_z^2}\cdot\sqrt{b_x^2+b_y^2+b_z^2}}\]
\subsection{两点距离}
空间2点的距离:
\[d=|M_1M_2|=\sqrt{(x_2-x_1)^2+(y_2-y_1)^2+(z_2-z_1)^2}\]
\subsection{点到平面的距离}
点$M_0(x_0,y_0,z_0)$到平面$\pi:Ax+By+Cz+D=0$的距离为
\[d=\frac{|Ax_0+By_0+Cz_0+D|}{\sqrt{A^2+B^2+C^2}}\]
\subsection{点到直线的距离}
点$M_0(x_0,y_0,z_0)$到平面$L_0:\frac{x-x_0}{l_0}+\frac{y-y_0}{m_0}+\frac{z-z_0}{n_0}$的距离为,其中$\overrightarrow{M_1P}$为直线$L_0$的法向量。
\[d=\frac{|\overrightarrow{M_1M_0}\times\overrightarrow{M_1P}|}{\overrightarrow{M_1P}}\]
\subsection{二次曲面方程}
\subsubsection{圆柱面}
\[x^2+y^2=R^2\]
\subsubsection{椭圆柱面}
\[\frac{x^2}{a^2}+\frac{y^2}{b^2}=1\]
\subsubsection{椭圆抛物面}
\[\frac{x^2}{a^2}+\frac{y^2}{b^2}=2pz\]
\subsubsection{椭球面}
\[\frac{x^2}{a^2}+\frac{y^2}{b^2}+\frac{z^2}{c^2}=1\]
\subsubsection{二次锥面}
\[\frac{x^2}{a^2}+\frac{y^2}{b^2}-\frac{z^2}{c^2}=0\]
\section{多元函数微分学}
\subsection{复合函数的微分}
设$z=f(u,v), u=\varphi(x,y), v=\phi(x,y)$,则
\begin{equation*}
\left\{%
\begin{array}{@{}l}    
    \frac{\partial z}{\partial x}=\frac{\partial z}{\partial u}\cdot\frac{\partial u}{\partial x}+\frac{\partial z}{\partial v}\cdot\frac{\partial v}{\partial x}\\[5pt]
    \frac{\partial z}{\partial y}=\frac{\partial z}{\partial u}\cdot\frac{\partial u}{\partial y}+\frac{\partial z}{\partial v}\cdot\frac{\partial v}{\partial y}\\
\end{array}
\right.
\end{equation*}

设$z=f(x,y,u,v), u=\varphi(x,y), v=\phi(x,y)$,则
\begin{equation*}
\left\{%
\begin{array}{@{}l}    
    \frac{\partial z}{\partial x}=\frac{\partial f}{\partial x}+\frac{\partial f}{\partial u}\cdot\frac{\partial u}{\partial x}+\frac{\partial f}{\partial v}\cdot\frac{\partial v}{\partial x}\\[5pt]
    \frac{\partial z}{\partial y}=\frac{\partial f}{\partial y}+\frac{\partial f}{\partial u}\cdot\frac{\partial u}{\partial y}+\frac{\partial f}{\partial v}\cdot\frac{\partial v}{\partial y}\\
\end{array}
\right.
\end{equation*}
\subsection{全微分公式}
\[dz=\frac{\partial z}{\partial x}dx+\frac{\partial z}{\partial y}dy
\quad du=\frac{\partial u}{\partial x}dx+\frac{\partial u}{\partial y}dy+\frac{\partial u}{\partial z}dz\]
\subsection{隐函数微分}
设$F(x,y)=0$,则
\[\frac{d y}{d x}=-\frac{F_x'(x,y)}{F_y'(x,y)}\]

设$F(x,y,z)=0$,则
\[\frac{\partial z}{\partial x}=--\frac{F_x'(x,y,z)}{F_z'(x,y,z)}\quad \frac{\partial z}{\partial y}=--\frac{F_y'(x,y,z)}{F_z'(x,y,z)}\]
\subsection{微分法在几何上的应用}
设曲面$F(x,y,z)=0$上有一点$M(x_0,y_0,z_0)$,则
过此点的法向量为
\[
\vec{n}=\left(F_{x}'\left(x_{0}, y_{0}, z_{0}\right), F_{y}'\left(x_{0}, y_{0}, z_{0}\right), F_{z}'\left(x_{0}, y_{0}, z_{0}\right)\right)
\]
过此点的切平面方程
\[
F_{x}'\left(x_{0}, y_{0}, z_{0}\right)\left(x-x_{0}\right)+F_{y}'\left(x_{0}, y_{0}, z_{0}\right)\left(y-y_{0}\right)+F_{z}'\left(x_{0}, y_{0}, z_{0}\right)\left(z-z_{0}\right)=0
\]
过此点的法线方程
\[
\frac{x-x_{0}}{F_x'\left(x_{0}, y_{0}, z_{0}\right)}=\frac{y-y_{0}}{F_{y}'\left(x_{0}, y_{0}, z_{0}\right)}=\frac{z-z_{0}}{F_{z}'\left(x_{0}, y_{0}, z_{0}\right)}
\]
\subsection{方向导数与梯度}
设三元函数$u=f(x,y,z)$在$M_0(x_0,y_0,z_0)$处可微,则$u$在点$M_0$沿任意方向$\vec{l}=(\cos\alpha,\cos\beta,\cos\gamma)$存在方向导数$\frac{\partial f}{\partial l}$,且有
\begin{align*}
    \left.\frac{\partial f}{\partial l}\right|_{M_0}&=
\left.\frac{\partial f}{\partial x}\right|_{M_0}\cos\alpha+
\left.\frac{\partial f}{\partial y}\right|_{M_0}\cos\beta+
\left.\frac{\partial f}{\partial z}\right|_{M_0}\cos\gamma\\
&=\nabla f_{M_0}\cdot\vec{l}
\end{align*}
其中$\nabla f_{M_0}$是函数$f(x,y,z)$在点$M_0$出的梯度,即梯度
\[\nabla f=\left(\frac{\partial f}{\partial x},\frac{\partial f}{\partial y},\frac{\partial f}{\partial z}\right)\]
\subsection{多元函数的极值}
\subsubsection{必要条件}
设函数$z=f(x,y)$在点$P(x_0,y_0)$存在一阶偏导数,且点$P$是函数$z$的极值点,则$f_x(x_0,y_0)=0,f_y(x_0,y_0)=0$
\subsubsection{充分条件} 
设函数$z=f(x,y)$在点$P(x_0,y_0)$存在二阶连续偏导数,且点$P$是函数$z$的极值点,则$f_x(x_0,y_0)=0,f_y(x_0,y_0)=0$ 。令$f_{xx}(x_0,y_0)=A,f_{xy}(x_0,y_0)=B,f_{yy}(x_0,y_0)=C$,则有
\begin{itemize}[nosep,left=\parindent,align=left]
    \item $AC-B^2>0$时,$A<0$为极大值,$A>0$为极小值
    \item $AC-B^2<0$时,无极值
    \item $AC-B^2=0$时,不确定
\end{itemize}
\section{多元函数积分学}
\subsection{重积分及其应用}
\subsubsection{曲面面积}
设曲面$S$由方程$z=f(x,y)$给出,$D_{xy}$为曲面$S$在$xOy$面上的投影区域,则曲面面积
\begin{align*}
    A&=\int_{D_{xy}}\sqrt{1+f_x^2(x,y)+f_y^2(x,y)}dxdy\\
     &=\int_{D_{yz}}\sqrt{1+g_y^2(x,y)+g_z^2(x,y)}dydz\\
     &=\int_{D_{zx}}\sqrt{1+h_z^2(x,y)+h_x^2(x,y)}dzdx
\end{align*}
\subsubsection{平面薄片的质心}
对于由$n$个质点组成的系统,其质心坐标$\bar{x}=\sum_{i=1}^{n}m_ix_i/\sum_{i=1}^{n}m_i$ ,$\bar{y}=\sum_{i=1}^{n}m_iy_i/\sum_{i=1}^{n}m_i$ 。设一平面薄片在$xOy$面的闭曲域$D$上连续,其面密度函数为$\rho(x,y)$ ,则其质心坐标为
\[
\bar{x}=\frac{\iint_{D} x \rho(x, y) d \sigma}{\iint_{D} \rho(x, y) d \sigma}\quad \bar{y}=\frac{\iint_{D} y \rho(x, y) d \sigma}{\iint_{D} \rho(x, y) d \sigma}
\]
\subsubsection{平面薄片的转动惯量}
质点系对于$x$轴和$y$轴的质量分别为$I_x=\sum_{i=1}^{n}y_i^2m_i$ ,$I_y=\sum_{i=1}^{n}x_i^2m_i$。对于平面薄片。有
\[
I_{x}=\iint_{D} y^{2} \rho(x, y) d \sigma \quad I_{y}=\iint_{D} x^{2} \rho(x, y) d \sigma
\]
\subsection{柱面坐标和球面坐标}
\subsubsection{柱面坐标}
坐标变换与积分元
\[
\left\{
\begin{array}{@{}l}
x=\rho\cos\varphi\\
y=\rho\sin\varphi\\
z=z\\
\end{array}\right.
\quad dS=\rho d\rho d\varphi \ (z\text{方向}) \quad dV=\rho d\rho d\varphi dz
\]
\begin{align*}
    \iiint_\Omega f(x,y,z)dxdydz 
    &=\iiint_\Omega F(\rho,\varphi,z)\rho d\rho d\varphi dz\\
    &=\iiint_\Omega f(\rho\cos\varphi,\rho\sin\varphi,z)\rho d\rho d\varphi dz
\end{align*}
\subsubsection{球面坐标}
坐标变换与积分元
\[
\left\{\begin{array}{@{}l}
x=r\sin \theta \cos \varphi\\
y=r\sin \theta \sin \varphi\\
z=r\cos \theta \\
\end{array}
\right.
\quad dS=r^2\sin \theta d\theta d\varphi \ (r\text{方向}) \quad dV=\rho d\rho d\varphi dz
\]
\begin{align*}
    \iiint_\Omega f(x,y,z)dxdydz 
    &=\iiint_\Omega F(r,\theta,\varphi) r^2\sin \theta dr d\theta d\varphi\\
    &=\iiint_\Omega f(r\sin \theta \cos \varphi,r\sin \theta \sin \varphi,r\cos \theta)r^2\sin \theta dr d\theta d\varphi
\end{align*}
\subsection{曲线积分}
\subsubsection{对弧长的曲线积分}
设$f(x,y)$在曲线弧$L$上连续且有定义,$L$的参数方程为
\[\left\{\begin{array}{@{}l}
    x=\varphi (t)\\
    y=\psi (t)\\
\end{array}
\right.,(\alpha\le t \le \beta)\]
且$\varphi (t), \psi (t)$在$[\alpha,\beta]$上具有一阶连续偏导数,则
\[\int_L f(x,y)ds=\int_\alpha^\beta f\big(\varphi (t),\psi (t)\big)\sqrt{\varphi'^2 (t)+\psi'^2 (t)}dt\ (\alpha<\beta)\]
\subsubsection[对坐标的曲线积分]{对坐标的曲线积分(求环流量)}
设$P(x,y), Q(x,y)$在曲线弧$L$上连续且有定义,$L$的参数方程同上,则
\[
\int_{L} P(x, y) d x+Q(x, y) d y=\int_{\alpha}^{\beta}\big(P[\varphi(t), \psi(t)] \varphi^{\prime}(t)+Q[\varphi(t), \psi(t)] \psi^{\prime}(t)\big) d t
\]

两类曲线积分的关系
\[
\int_{L} P d x+Q d y=\int_{L}(P \cos \alpha+Q \cos \beta) d s
\]
\subsection{曲面积分}
\subsubsection{对面积的曲面积分}
设积分曲面$\Sigma$由方程$z=z(x,y)$给出,则被积函数$f(x,y,z)$在曲面$\Sigma$上对面积的积分为
\[\iint_{\Sigma} f(x, y, z) d s=\iint_{D_{w}} f[x, y, z(x, y)] \sqrt{1+z_{x}^{2}(x, y)+z_{y}^{2}(x, y)} d x d y\]
\subsubsection[对坐标的曲面积分]{对坐标的曲面积分(求通量)}
当取曲面的上、前、右侧时取正号。
\begin{align*}
    \iint_{\Sigma} R(x, y, z) d x d y &=\pm \iint_{D_{xy}} R[x, y, z(x, y)] d x d y \\ 
    \iint_{\Sigma} P(x, y, z) d y d z &=\pm \iint_{D_{yz}} P[x(y, z), y, z] d y d z \\ 
    \iint_{\Sigma} Q(x, y, z) d z d x &=\pm \iint_{D_{zx}} Q[x, y(z, x), z] d z d x
\end{align*}
\subsection{格林公式}
在一重积分中,牛顿-莱布尼茨公式
\[\int_a^b F'(x)dx=F(b)-F(a)\]
反映了函数在区间的积分与其原函数在区间端点值的情况,在二重积分中,有类似的格林公式。

设闭曲线$D$由分段光滑的曲线$L$围成,$L$是$D$取正向的边界曲线,函数$P(x,y),Q(x,y)$在$D$上具有一阶连续偏导数,则有
\[\iint_D\left(\frac{\partial Q}{\partial x}-\frac{\partial P}{\partial y}\right)dxdy=\oint_LP\,dx+Q\,dy\]
\subsection{高斯公式}
设空间闭区域$\Omega$是由分片光滑的闭曲面$\Sigma$围成,若函数$P(x,y,z)$,$Q(x,y,z)$,$R(x,y,z)$在$\Omega$上具有一节连续偏导数,则有
\begin{align*}
    \iiint_\Omega\left(\frac{\partial P}{\partial x}+\frac{\partial Q}{\partial y}+\frac{\partial R}{\partial x}\right)dV &=\oiint_\Sigma Pdydz+Qdzdx+Rdxdy\\
    &=\oiint_\Sigma \left(P\cos\alpha+Q\cos\beta+R\cos\gamma\right)dS
\end{align*}

这里$\Sigma$时整个边界曲面$\Omega$的外侧,$\cos\alpha,\cos\beta,\cos\gamma$是$\Sigma$在点$(x,y,z)$处的法向量的方向余弦。高斯公式揭示了通量与散度的关系。设向量场
\[\vec{A}(x,y,z)=P(x,y,z)\vec{i}+Q(x,y,z)\vec{j}+R(x,y,z)\vec{k}\]
则向量场$\vec{A}$通过曲面$\Sigma$向着指定侧的通量为积分
\[\iint_\Sigma\vec{A}\cdot\vec{n}\,dS\]
将向量场$\vec{A}$的散度记作$\nabla\cdot \vec{A}$,有
\[\nabla\cdot\vec{A}=\frac{\partial P}{\partial x}+\frac{\partial Q}{\partial y}+\frac{\partial R}{\partial z}\]
利用向量场的通量和散度的关系,高斯公式可以写成向量形式
\[\iiint_\Omega\nabla\cdot\vec{A}\,dV=\oiint_\Sigma A_n\,dS\]
\subsection{斯托克斯公式 —— 曲线积分与曲面积分的关系}
设$\Gamma$为分段光滑的有向闭曲线,$\Sigma$是以$\Gamma$为边界的分片光滑的有向曲面,$\Gamma$的侧与$\Sigma$的正向符合右手规则,若函数$P(x,y,z)$,$Q(x,y,z)$,$R(x,y,z)$在曲线$\Sigma$上具有一阶连续偏导数,则有
\begin{multline*}
\iint_\Sigma\left(\frac{\partial R}{\partial y}-\frac{\partial Q}{\partial z}\right)dydz+
\left(\frac{\partial P}{\partial z}-\frac{\partial R}{\partial x}\right)dzdx+
\left(\frac{\partial Q}{\partial x}-\frac{\partial P}{\partial y}\right)dxdy\\
=\oint_\Gamma Pdx+Qdy+Rdz
\end{multline*}
斯托克斯公式揭示了环流量与旋度的关系。设向量场
\[\vec{A}(x,y,z)=P(x,y,z)\vec{i}+Q(x,y,z)\vec{j}+R(x,y,z)\vec{k}\]
其中函数$P,Q,R$均连续,$\Gamma$是$A$的定义域里面一条分段光滑的有向闭曲线,$\vec{\tau}$是$\Gamma$在点$(x,y,z)$处的单位切向量,则积分
\[\oint_\Gamma \vec{A}\cdot\vec{\tau}\,ds\]
为向量场$A$沿有向闭曲线$\Gamma$的环流量。将向量场$\vec{A}$的旋度记作$\nabla\times \vec{A}$,有
\begin{align*}
\nabla\times\vec{A} &=
\left|
\begin{array}{@{}ccc@{}}
\vec{i}&\vec{j}&\vec{k}\\
\frac{\partial }{\partial x}&\frac{\partial }{\partial y}&\frac{\partial }{\partial z}\\[2pt]
    P&Q&R\\
\end{array}
\right|\\
 &=
\left(\frac{\partial R}{\partial y}-\frac{\partial Q}{\partial z}\right)\vec{i}+
\left(\frac{\partial P}{\partial z}-\frac{\partial R}{\partial x}\right)\vec{j}+
\left(\frac{\partial Q}{\partial x}-\frac{\partial P}{\partial y}\right)\vec{k}
\end{align*}
利用环流量与旋度的关系,斯托克斯公式可以写成向量形式
\[\iint_\Sigma\nabla\times\vec{A}\cdot\vec{n}\,dS=\oint_\Gamma\vec{A}\cdot\vec{\tau}\,ds\]
\section{无穷级数}
\subsection{常数项级数}
等比数列:
\[1+q+q^2+\cdots+q^{n-1}=\frac{1-q^{n}}{1-q}\]

等差数列:
\[1+2+3+\cdots+n=\frac{n(n+1)}{2}\]

平方和数列:
\[1^2+2^2+3^2+\cdots+n^2=\frac{n(n+1)(2n+1)}{6}\]

立方和数列:
\[1^3+2^3+3^3+\cdots+n^3=\frac{n^2(n+1)^2}{4}\]
\subsection{正项级数审敛法}
\subsubsection{比较审敛法}
设$A:\sum_{n=1}^\infty u_n$和$B:\sum_{n=1}^\infty v_n$都是正项级数,令
\[\lim_{n\to\infty}\frac{u_n}{v_n}=l\]
收敛情况如下
\begin{itemize}[nosep,left=2em]
    \item $0\le l \le \infty$,且级数$B$收敛,那么级数$A$收敛。
    \item $0<l$或者$l=+\infty$,且级数$B$发散,那么级数$A$发散。
\end{itemize}
\subsubsection{比值审敛法}
设$\sum_{n=1}^\infty u_n$为正项级数,令
\[\lim_{n\to\infty}\frac{u_{n+1}}{u_n}=\rho\]
收敛情况如下
\begin{itemize}[nosep,left=2em]
    \item $\rho<1$\ 级数收敛
    \item $\rho>1 \ (\lim_{n\to\infty}{u_{n+1}}/{u_n}=\infty)$\  级数发散
    \item $\rho=1$\ 不确定
\end{itemize}
\subsubsection{根值审敛法}
设$\sum_{n=1}^\infty u_n$为正项级数,令
\[\lim_{n\to\infty}\sqrt[n]{u_n}=\rho\]
收敛情况如下
\begin{itemize}[nosep,left=2em]
    \item $\rho<1$\ 级数收敛
    \item $\rho>1 \ (\lim_{n\to\infty}\sqrt[n]{u_n}=\infty)$\  级数发散
    \item $\rho=1$\ 不确定
\end{itemize}
\subsection{交错级数审敛法}
设$\sum_{n=1}^\infty (-1)^{n-1}u_n$为交错级数,满足条件
\[u_n\ge u_{n+1}\]
\[\lim_{n\to\infty}u_n=0\]
% \[
% \begin{cases}
% u_n\ge u_{n+1}\\
% \lim_{n\to\infty}u_n=0
% \end{cases}
% \]
那么级数收敛。
\subsection{绝对收敛与条件收敛}
设级数$A$,$B$分别为
\[A:\sum_{n=1}^\infty u_n\quad B:\sum_{n=1}^\infty |u_n|\]
那么有
\begin{itemize}[nosep,left=2em]
    \item 若级数$B$收敛,则称级数$A${\bfseries 绝对收敛}
    \item 若级数$A$收敛,而级数$B$发散,则称级数$A${\bfseries 条件收敛}
    \item 若级数$A$绝对收敛,那么$A$必定收敛
\end{itemize}
\subsection{幂级数}
如果给定一个定义在区间$I$上的函数列
\[u_1(x),u_2(x),u_3(x),\cdots,u_1(x),\cdots\]
那么由函数列构成的表达式
\[u_1(x)+u_2(x)+u_3(x)+\cdots+u_1(x)+\cdots\]
称为定义在区间$I$上的函数项无穷级数,函数项级数。

对于一个确定的$x_0\in I$,这个级数有可能收敛也有可能发散。如果$x_0$使这个级数收敛,则称它为函数项级数的收敛点,否则称为发散点。由全体收敛点构成的集合称为函数项级数的收敛域,由全体发散点构成的集合称为函数项级数的发散域。

函数项级数中最简单常见的一类就是幂级数,它的形式为
\[A:\sum_{n=1}^\infty a_n x^n=a_0+a_1x+a_2x^2+\cdots+a_nx^n+\cdots\]
如果幂级数$A$不是仅在$x=0$一点收敛,也不是在整个数轴上都收敛,那么必定存在一个确定的正数$R$使得
\begin{itemize}[nosep,left=2em]
    \item $|x|<R$ 时,幂级数绝对收敛
    \item $|x|>R$ 时,幂级数发散
    \item $|x|=\pm R$ 时,幂级数可能收敛,也可能发散
\end{itemize}
正数$R$叫做幂级数的收敛半径。$R$可以由下面方法得来
\[\text{令}\lim_{n\to\infty}\left| \frac{a_{n+1}}{a_n}\right|=\rho\ ,\quad\text{则}\ %
R=\begin{cases}
  1/\rho,&\rho\ne0\\
  +\infty,&\rho=0\\
  0,&\rho=+\infty  
\end{cases}\]
\subsection{函数展开成幂级数}
函数$f(x)$在$x_0$的邻域$U(x_0)$具有各阶导数,展开为泰勒级数
\[f(x)=\sum_{n=0}^\infty \frac{1}{n!}f^{(n)}(x_0)(x-x_0)^n,x\in U(x_0)\]
的充分必要条件是$f(x)$在该邻域内的泰勒公式的余项$R_n$的极限为零,即

\[\lim_{n\to\infty}R_n=\lim_{n\to\infty}\frac{f^{(n+1)}(\xi)}{(n+1) !}\left(x-x_{0}\right)^{n+1}=0\]

\subsection{常见函数展开成幂级数}
\begin{gather*}
    \frac{1}{1-x} =1+x+x^2+\cdots+x^n+\cdots=\sum_{n=0}^\infty x^n,x\in(-1,1)\\
    e^x=1+x+\frac{x^2}{2!}+\cdots+\frac{x^n}{n!}+\cdots=\sum_{n=0}^\infty \frac{x^n}{n!},x\in(-\infty,+\infty)\\
    \sin x=x-\frac{x^3}{3!}+\cdots+(-1)^n\frac{x^{2n+1}}{(2n+1)!}+\cdots=\sum_{n=0}^\infty (-1)^n\frac{x^{2n+1}}{(2n+1)!},x\in(-\infty,+\infty)\\
    \ln(1+x)=x-\frac{x^2}{2}+\frac{u^3}{3}-\cdots+(-1)^n\frac{x^{n+1}}{n+1}+\cdots=\sum_{n=0}^\infty (-1)^n\frac{x^{n+1}}{n+1},x\in(-1,+1)\\
    (1+x)^a=1+ax+\frac{a(a-1)}{2!}x^2+\cdots+\frac{a(a-1)\cdots(a-n+1)}{n!}x^n+\cdots,x\in(-1,1)
\end{gather*}

\subsection{欧拉公式}
\[e^ix=\cos x+i\sin x\quad \cos x=\frac{e^{ix}+e^{-ix}}{2}\quad \sin x=\frac{e^{ix}-e^{-ix}}{2i}\]
\subsection{三角级数}
设由正弦级数组成的函数$f(t)$为
\[f(t)=A_0+\sum_{n=1}^\infty A_n\sin (n\omega t+\varphi_n)\]
其中$A_0,A_n,\varphi_n$为常数,将正弦函数$A_n\sin (n\omega t+\varphi_n)$展开得
\[A_n\sin (n\omega t+\varphi_n)=A_n\sin n\omega t\cos \varphi_n+A_n\cos n\omega t\sin \varphi_n\]
令$a_0/2=A_0$, $a_n=A_n\sin \varphi_n$, $b_n=A_n\cos \varphi_n$,
$\omega=\pi/(2l)$, $\pi t/l=x$得到三角级数
\[f(t)=\frac{a_0}{2}+\sum_{n=1}^\infty \left(a_n\cos nx+b_n\sin nx \right)\]
\subsection{傅立叶级数}
设$f(x)$是以周期为$2\pi$的周期函数,且能展开成三角级数,即
\[f(x)=\frac{a_0}{2}+\sum_{n=1}^\infty \left(a_n\cos nx+b_n\sin nx \right)\]
对等式两边在区间$[-\pi,\pi]$逐项积分
\[\int_{-\pi}^\pi f(x)dx=\frac{a_0}{2}+\sum_{n=1}^\infty \left(a_n
\int_{-\pi}^\pi \cos nx\, dx+b_n \int_{-\pi}^\pi \sin nx\, dx\right)\]
由三角函数的正交性,在积分式两边依次乘$\cos nx$和$\sin nx$可求得
\begin{align*}
    a_n&=\frac 1\pi \int_{-\pi}^\pi f(x)\cos nx\, dx\ (n=0,1,2,\ldots)\\
    b_n&=\frac 1\pi \int_{-\pi}^\pi f(x)\sin nx\, dx\ (n=1,2,3,\ldots)
\end{align*}
当$a_n$,$b_n$满足上述关系时,三角级数
\[\frac{a_0}{2}+\sum_{n=1}^\infty \left(a_n\cos nx+b_n\sin nx \right)\]
叫做函数$f(x)$的傅里叶级数
\subsection[周期函数的傅立叶级数]{周期为2$l$的周期函数的傅立叶级数}
若$f(x)$是以周期不是$2\pi$,而是以$2l$为周期的周期函数,且在$[-l,l]$上可积,则
\begin{align*}
    a_n&=\frac 1l \int_{-l}^l f(x)\cos \frac{n\pi}{l}x\, dx\ (n=0,1,2,\ldots)\\
    b_n&=\frac 1l \int_{-l}^l f(x)\sin \frac{n\pi}{l}x\, dx\ (n=1,2,3,\ldots)
\end{align*}
\section{常微分方程}
\subsection{微分方程的相关概念}
一般地,表示未知函数、未知函数的导数与自变量间的关系的方程,叫做微分方程。微分方程中出现的未知函数的最高阶导数的结束叫做微分方程的阶。$n$阶微分方程的一般形式是
\[F\left(x,y,y',y'',\cdots,y^{(n)}\right)=0\]
需要指出,方程中$y^{(n)}$是必须出现的,其他变量不一定要出现。

设函数$y=\varphi(x)$在区间$I$上具有$n$阶连续偏导数,如果在区间$I$上$y=
\varphi(x)$满足方程
\[F\left(x,\varphi(x),\varphi(x)',\varphi(x)'',\cdots,\varphi(x)^{(n)}\right)\equiv0\]
则称$y=\varphi(x)$是微分方程的解。如果微分方程的解含有任意常数,且任意常数的个数与微分方程的阶数相同\footnote{任意常数必须相互独立},这样的解叫做微分方程的通解。给定初始条件,确定了通解中的所有任意常数后,就得到微分方程的特解。
\subsection{一阶线性微分方程}
一阶线性微分方程,如
\[y'+p(x)y=q(x)\]
它对于未知函数及其导数来讲都是一次方程,即线性的。如果方程的$q(x)=0$,那么方程就是齐次的,否则就是非齐次的。
\[\begin{cases}
q(x)=0&,y=Ce^{-\int p(x)dx}\\
q(x)\ne0&,y=e^{-\int p(x)dx}\left(\int q(x)e^{\int p(x)}dx+C\right)\\
\end{cases}\]
\subsection{二阶常系数齐次线性微分方程}
对于二阶齐次线性微分方程
\[y''+py'+qy=0\]
如果其中$p$,$q$均为常数,则称方程为二阶常系数齐次线性微分方程,若$p$,$q$不全为常数,则称其为二阶变系数齐次线性微分方程。二阶常系数齐次线性微分方程的通解求法如下
\begin{enumerate}[nosep,leftmargin=0pt,labelindent=2em,itemindent=*]
    \item 写出微分方程的特征方程$r^2+pr+q=0$
    \item 求出特征方程的两个根$r_1, r_2$
    \item 根据表 \ref{tab:differential_equations} 写出通解。
\end{enumerate}
\begin{table}[htbp]
  \centering
  \caption{微分方程通解求法}
  \small
    \begin{tabular}{cc}
    \toprule
    特征方程的根$r_1,r_2$   & 微分方程通解 \\
    \midrule
   不等实根$r_1\ne r_2$ & $y=C_1e^{r_1x}+C_2e^{r_2x}$\\
   相等实根$r_1 = r_2$  & $y=\left(C_1+C_2 x\right)e^{r_1x}$\\
   共轭复根$r_{1,2} = \alpha \pm \beta i$ & $y=e^{\alpha x}\left(C_1\cos\beta x +C_2\sin \beta x\right)$\\
    \bottomrule
    \end{tabular}%
  \label{tab:differential_equations}%
\end{table}%
\subsection{二阶常系数非齐次线性微分方程}
二阶常系数非齐次线性微分方程的一般形式为
\[y''+py'+qy=f(x)\]
其通解可以归结为对应的齐次方程$y''+py'+qy=0$的通解和其本身的一个特解。具体解法参考第 \ref{sec:n_differential_equations} 节。
\subsection[高阶常系数线性微分方程]{$n$阶常系数线性微分方程}
齐次方程的通解$y_h(x)$形式
\label{sec:n_differential_equations}
\begin{table}[htbp]
  \centering
  \caption{不同特征根对应的通解}
   \small
    \begin{tabular}{cc}
    \toprule
    特征根 $r$   & 通解 $y_h(x)$  \\
    \midrule
    单实根 &  $e^{rx}$   \\
    $r$重实根 & $\left(C_{r-1}x^{r-1}+C_{r-2}x^{r-2}+\cdots+C_1x+C_0\right)e^{rx}$    \\
    共轭复根$r_{1,2}=\alpha\pm\beta j$ & $e^{\alpha x}[C\cos\beta x+D\sin \beta x]$\\
    $r$重共轭复根 & $\left(A_{r-1}x^{r-1}\cos(\beta x+\theta_{r-1})+\cdots+A_0\cos(\beta x+\theta_{0})\right)$    \\
    \bottomrule
    \end{tabular}%
  % \label{tab:addlabel}%
\end{table}%

非齐次方程的特解$y_p(x)$形式
% Table generated by Excel2LaTeX from sheet 'Sheet1'
\begin{table}[htbp]
  \centering
  \caption{不同激励对应的特解}
   \small
    \begin{tabular}{ccc}
    \toprule
    激励 $f(x)$   & \multicolumn{2}{c}{特解 $y_p(x)$}  \\
    \midrule
    \multirow{2}*{\mbox{$x^m$}} &  $P_mx^m+P_{m-1}x^{m-1}+\cdots+P_1x+P_0$   & 所有特征根不相等\\
          &  $x^r\left(P_mx^m+P_{m-1}x^{m-1}+\cdots+P_1x+P_0\right)$              & 有$r$重相等特征根\\
    \midrule
    \multirow{3}*{$e^{\alpha x}$} & $Pe^{\alpha x}$                     & $\alpha$  不等于特征根\\
          & $(P_1x+P_0)e^{\alpha x}$                                    & $\alpha$等于特征单根 \\
          & $P_rx^r+P_{r_1}x^{r_1}+\cdots+P_1x+P_0$                     & $\alpha$等于$r$重特征单根 \\
    \midrule
    $\cos(\beta x)$或$\sin(\beta x)$ & $ P\cos\beta x+Q\sin \beta x$   & 所有特征根不相等 \\
    \bottomrule
    \end{tabular}%
  % \label{tab:addlabel}%
\end{table}%

\chapter{线性代数部分}
\section{行列式}
\section{矩阵}
\section{向量}
\section{线性方程组}
\section{矩阵的特征值和特征向量}
\section{相似矩阵和二次型}

\chapter{概率论部分}
\section{随机事件及其概率}
\subsection{概率的定义及其计算}
\subsection{随机变量及其分布}
\subsection{离散型随机变量}
\subsection{连续型随机变量}
\section{一维随机变量及其分布}
\section{多维随机变量及其分布}
\section{大数定律和中心极限定理}
\section{数理统计}
\section{随机变量的数字特征}
\subsection{数学期望}
离散型随机变量
\[E(X)=\sum_{k=1}^{+\infty}x_kp_k\]

连续性随机变量
\[E(X)=\int_{-\infty}^{+\infty}xf(x)dx\]
\subsection{方差}
\[D\big(X\big)=E\Big(\big(X-E(X)\big)^2\Big)=E\big(X^2\big)-E^2\big(X\big)\]
\subsection{协方差}
\begin{align*}
    cov\big(X,Y\big)={}& E\Big(\big(x-E(x)\big)\big(Y-E(Y)\big)\Big)\\
                    ={}& E(XY)-E(X)E(Y)\\
                    ={}& \pm\frac 12\big(D(X\pm Y)-D(X)-D(Y)\big)
\end{align*}

\subsection{相关系数}
\[\rho_{XY}=\frac{cov(X,Y)}{\sqrt{D(X)D(Y)}}\]
\end{document}


%\usepackage{fancyhdr}%页面风格设置
%\pagestyle{fancy}{% bfseries
%    \lhead{}% 页眉
%    \chead{}
%    \rhead{}
%    \lfoot{}% 页脚
%    \cfoot{-{\thepage}-\pageref{total}\,-}%第几页共几页
%    \rfoot{}
%}%修改格式
%\renewcommand{\headrulewidth}{0pt}%默认0.4pt
%\renewcommand{\footrulewidth}{0pt}

\pagestyle{plain}

\usepackage[nottoc]{tocbibind}%增加目录内容,用[nottoc]可以取消输出目录本身
\usepackage[section]{placeins}%避免浮动体跨过\section等章节标题
\usepackage[breaklinks]{hyperref}%%目录引用等超链接
%\hypersetup{linkcolor=blue,
%            urlcolor=blue,
%            citecolor=blue,
%            bookmarksnumbered=true,
%            bookmarksopen=true
%            }
\hypersetup{%目录引用等超链接
    %colorlinks=true,
    %bookmarks=true,
    bookmarksopen=true,
    pdfborder=0 0 0,  %使超链接的框不显示d
}

te
% Table generated by Excel2LaTeX from sheet 'Sheet1'
\begin{table}[htbp]
  \centering
  \caption{Add caption}
    \begin{tabular}{cc}
    \toprule
    激励    & 特解 \\
    \midrule
    \multirow{2}[1]{*}{x} & 1 \\
          & 2 \\
    \multirow{3}[0]{*}{y} & 1 \\
          & 2 \\
          & 3 \\
    \multirow{2}[1]{*}{z} & 1 \\
          & 2 \\
    \bottomrule
    \end{tabular}%
  \label{tab:addlabel}%
\end{table}%
